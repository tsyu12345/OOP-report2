\documentclass[dvipdfmx]{jsarticle}
\usepackage[T1]{fontenc}
\usepackage[dvipdfmx]{hyperref}
\usepackage{lmodern}
\usepackage{latexsym}
\usepackage{amsfonts}
\usepackage{amssymb}
\usepackage{mathtools}
\usepackage{nccmath}
\usepackage{amsthm}
\usepackage{multirow}
\usepackage{graphicx}
\usepackage{wrapfig}
\usepackage{here}
\usepackage{float}
\usepackage{ascmac}
\usepackage{url}

\title{オブジェクト指向の考えを利用したJavaのペア・プログラミング課題}
\author{文理学部情報科学科\\5419045 高林 秀 出川慎吾}
\date{\today}

\begin{document}

\maketitle

\begin{abstract}
  本稿では、今年度オブジェクト指向プログラミングの課題研究2として、前回の課題研究で制作したJavaプログラムを、パ−トナーのソースコードと統合し、加えてZoog以外の新しいキャラクターの描画を行うものである。本演習には、Javaを使用した。
  結果は、、、で、、、であった。
\end{abstract}

\section{目的}
本稿は、今年度オブジェクト指向プログラミングの課題研究として、Javaのペア・プログラミングを行うものである。今回の課題は、前回の課題研究1で作成したZoogクラスを、パートナーのJavaソースコードと統合すること、並びにZoogとは異なる形、動作をする新しいキャラクターを描画すること、である。その際、Zoogと同一のinterface等を利用し、呼び出し側を共通化する。

\section{課題概要}
本課題は下記に示す、2種類の課題から構成されている。以下にそれぞれの詳細を示す。
\begin{enumerate}
  \item 課題研究1で制作したプログラムを1つのプログラムに統合する。\par
  自分とパートナーの課題研究1で作成したZoogプログラムを比較し、1つのプログラムで呼び出しが行えるよう統合する。その際、共通の抽象クラスやinterfaceを利用するなど、適切な継承関係を与えること。
  \item 新しいキャラクターを追加する。\par
  Zoogと見た目、動き方、攻撃方法(なにをすると動きが止まるか)が異なる新しいキャラクターを描画する。その際、1で使用した抽象クラスやinterfaceを利用するなどし、呼び出し側を共通化することが望まれる。
\end{enumerate}
\end{document}
